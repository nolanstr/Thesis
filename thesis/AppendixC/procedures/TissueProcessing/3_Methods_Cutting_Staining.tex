\section{Cutting}\label{Cutting}
After the plastic has cured, remove the specimen to be cut and use the microtome to shave away thin layers to be used for TEM.  

\section{Grid Staining}
Once thin sections have been placed on grids from section \ref{Cutting} the grids will need to be stained to increase the contrast for TEM.  Two chemicals will be Uranyl Acetate and Lead Citrate.

\subsection{Preparation}
Using the square petri-dish and wax from the cupboard cut the wax to fit the inside the petri-dish.  Clean the wax with alcohol and DI water to remove any impurities on the wax that would alter the grid samples.  This will also prevent the drops from coagulating together on the wax.  Simply rinse the wax to clean it off.

\subsection{Chemical Prep}
After the wax has been cleaned and cut remove the saturated Uranyl Acetate and Reynold's Lead Citrate from the refrigerator.  Grab two small 1 ml syringes from the drawer and fill up each syringe with either UA or Lead Citrate.  Then place one small filter on the end of the syringe filled with UA and two filters on the syringe filled with Lead Citrate.  

\subsection{UA Stain}\label{UA_Stain}
Using the 1 ml syringe with a single filter place a droplet of UA for each grid that you need to stain evenly spaced on the wax pad.  Use the forceps and remove the grids from the grid holder and place on top of the UA droplet.  Be sure to place the grid shiny side down to allow the UA to stain the specimen.

\subsection{Timer - 18 minutes}
Set the timer for 18 minutes.  During this time fill up enough 30 ml syringes with DI water for rinsing both UA and Lead Citrate.  You will need approximately 10 ml per sample per rinse.  Place a large filter on the end of the syringe.

\subsection{Staging Area}
Grab a small round petri dish and insert two filter papers to absorb the water following the rinse.  Use a pen or pencil to mark the paper to help organize the order of specimens to prevent a mix up.  

\subsection{First Rinse}\label{first_rinse}
After 18 minutes, pick up the grid with forceps and rinse with 10 ml of DI water.  Hold the forceps at a 60$^\circ$ angle from the horizontal and drip the water down the curved section of the forceps.  After the rinse, place the specimens inside the round petri dish to remove excess DI water.  Once all of the specimens have been placed on the filter paper, a few sodium hydroxide crystals will need to be placed inside the square petri dish.  The NaOH will help prevent any sort of moisture from interfering with the grid during the staining process.  Next, use the other 1 ml syringe with Lead Citrate and place drops on the wax pad following the same procedure mentioned before in section~\ref{UA_Stain}.

\subsection{Lead Citrate Rinse}
Using the forceps, grip the grid and place it on top of the Lead Citrate droplet with the shiny side down which allows the grid to be stained.  Set the timer for eight minutes.

\subsection{Second Rinse}
After eight minutes have passed, repeat the same step as in~\ref{first_rinse}.  Once the grids have completely dried, place them back in the grid holder and they are ready for the TEM.

\subsection{Cleanup}
Dispose of the petri dish in the unwanted UA container.