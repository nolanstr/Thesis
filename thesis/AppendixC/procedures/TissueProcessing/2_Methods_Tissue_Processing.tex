\subsection{Sorting}\label{sorting}
Begin first by sorting the tissue in two piles of tissue that was peeled and
tissue that was adjacent to the peeled region.  Then write down the
identification ID \# on the paper to keep the proper vial straight during the
tissue process.  \\

\subsubsection{Identification ID \#}\label{identification}
Sheep \#, L/R, E/P, P/A\\

For example, {\it UL-15A-B Left Equator Peel} can be reduced to {\it UL15LEP}

\section{Dehydration}\label{dehydration}
First place samples in glass vials.  Use forceps if it is required to remove
excess waste from the container.  Properly label the samples from before
section~\ref{identification} and place the label on the vial.  Before adhering
the label to the vial, write down the number of specimens in the vial to ensure
that the specimens don't get lost during the process.  Use tape to ensure that
the label will not be removed from the vial during the process.

\subsection{Buffer Rinse}\label{buffer rinse}
Remove the fixative from the existing vial using the micropipette.  Be sure not
to suck out the tissue.  Then fill the vial with buffer - 0.1M Sodium
Cacodylate buffer.

\subsubsection{Agitation}\label{agitation 5}
Put the sample vials in the rotating agitator for 5 minutes.  

\subsection{Buffer Rinse}\label{buffer rinse 2}
Remove the buffer from section~\ref{buffer rinse} and replace with new buffer -
0.1M Sodium Cacodylate buffer.

\subsubsection{Agitation}\label{agitation 5 number 2}
Put the sample vials in the rotating agitator again for 5 minutes. 

\subsection{Osmium dilution}\label{osmium dilution}
During the previous agitation step in section~\ref{agitation 5 number 2} dilute
the osmium tetroxide $OsO_4$ (4\% in $dH_2O$) with 0.2 M Sodium Cacodylate
buffer in a 1:1 mixture.  Be sure to filter the Osmium tetroxide with a
millipore filter to remove any excess particulate that would otherwise result
in artifacts inside the tissue.  

\subsection{Osmium rinse}\label{osmium rinse}
Remove the 0.1M Sodium cacadylate buffer from the vials and replace with the
diluted Osmium from section~\ref{osmium dilution}.  Use just enough diluted
Osmium to cover the tissue.

\subsubsection{Agitation}\label{agitation hr}
Put the sample vials back in the rotating agitator again for one hour. 

\subsection{DI water rinse}\label{DI water rinse}
Remove the diluted Osmium tetroxide from the vials and rinse with DI water.
The DI water will be filtered \footnote{\label{note1}The millipore filter is
used to remove any excess particulate that would otherwise result in artifacts
inside the tissue.}.  This step is done to remove excess osmium.

\subsubsection{Agitation}\label{agitation 5 number 3}
Put the sample vials back in the rotating agitator again for 5 minutes. 

\subsection{Uranyl Acetate rinse}\label{Uranyl Acetate rinse}
Remove the DI water from the vials and replace with Saturated 4\% Aqueous
Uranyl Acetate.  The Uranyl Acetate also needs to be filtered using a millipore
filter\footnoteref{note1} on a 10 ml syringe.  

\subsubsection{Agitation}\label{agitation 5 number 4}
Put the sample vials back in the rotating agitator again for one hour.  

\section{Final acetone dehydration step}
The final step of the dehydration process is to replace all of the moisture in
the tissue from $H_2O$ to pure acetone.  This is done with a series of rinses
in various percentages of alcohol with the last set of rinses in acetone.
**Note - if there is not enough alcohol mixtures in the hood then you will need
to make more.  When making the dilutions, use the graduated cylinder that is in
the sink and mix the highest concentrations first to ensure that the
percentages of alcohol is correct.  Start with 95 then 70 then 50 etc.  Also be
sure that the ethanol containers are covered to prevent evaporation during each
step of the dehydration process.

\subsection{50\% Ethanol Alcohol}
Remove the urinal acetate from the vial in section~\ref{Uranyl Acetate rinse}
to the appropriate container.  Next use the micropipette and fill the vial with
50\% Ethanol Alcohol; ensure that the tissue specimen is well covered.

\subsubsection{Agitation}
Put the sample vials in the rotating agitator again for 10 minutes. 

\subsection{70\% Ethanol Alcohol}
Remove the 50\% Ethanol Alcohol from the vial.  Next use the micropipette and
fill the vial with 70\% Ethanol Alcohol; ensure that the tissue specimen is
well covered.

\subsubsection{Agitation}
Put the sample vials in the rotating agitator again for 10 minutes. 

\subsection{95\% Ethanol Alcohol}
Remove the 70\% Ethanol Alcohol from the vial.  Next use the micropipette and
fill the vial with 95\% Ethanol Alcohol; ensure that the tissue specimen is
well covered.

\subsubsection{Agitation}
Put the sample vials in the rotating agitator again for 10 minutes. 

\subsection{95\% Ethanol Alcohol}
Remove the 95\% Ethanol Alcohol from the vial.  Next use the micropipette and
fill the vial with 95\% Ethanol Alcohol; ensure that the tissue specimen is
well covered.

\subsubsection{Agitation}
Put the sample vials in the rotating agitator again for 10 minutes. 

\subsection{100\% Ethanol Alcohol}
Remove the 95\% Ethanol Alcohol from the vial.  Next use the micropipette and
fill the vial with 100\% Ethanol Alcohol; ensure that the tissue specimen is
well covered.

\subsubsection{Agitation}
Put the sample vials in the rotating agitator again for 10 minutes. 

\subsection{100\% Ethanol Alcohol}
Remove the 100\% Ethanol Alcohol from the vial.  Next use the micropipette and
fill the vial with 100\% Ethanol Alcohol; ensure that the tissue specimen is
well covered.

\subsubsection{Agitation}
Put the sample vials in the rotating agitator again for 10 minutes. 

\subsection{100\% Ethanol Alcohol}
Remove the 100\% Ethanol Alcohol from the vial.  Next use the micropipette and
fill the vial with 100\% Ethanol Alcohol; ensure that the tissue specimen is
well covered.

\subsubsection{Agitation}
Put the sample vials in the rotating agitator again for 10 minutes. 

\subsection{100\% Ethanol Alcohol}
Remove the 100\% Ethanol Alcohol from the vial.  Next use the micropipette and
fill the vial with 100\% Ethanol Alcohol; ensure that the tissue specimen is
well covered.

\subsubsection{Agitation}
Put the sample vials in the rotating agitator again for 10 minutes. 

\subsection{Acetone}
Remove the 100\% Ethanol Alcohol from the vial.  Next use the micropipette and
fill the vial with acetone; ensure that the tissue specimen is well covered.

\subsubsection{Agitation}
Put the sample vials in the rotating agitator again for 10 minutes. 

\subsection{Acetone}
Remove the acetone from the vial.  Next use the micropipette and fill the vial
with acetone; ensure that the tissue specimen is well covered.

\subsubsection{Agitation}
Put the sample vials in the rotating agitator again for 10 minutes. 

\subsection{Acetone}
Remove the acetone from the vial.  Next use the micropipette and fill the vial
with acetone; ensure that the tissue specimen is well covered.

\subsubsection{Agitation}
Put the sample vials in the rotating agitator again for 10 minutes. 

\subsection{Acetone}
Remove the acetone from the vial.  Next use the micropipette and fill the vial
with acetone; ensure that the tissue specimen is well covered.

\subsubsection{Agitation}
Put the sample vials in the rotating agitator again for 10 minutes.

\section{Infiltration}\label{infiltration}
Once the tissue samples have been completely dehydrated and all moisture in the
sample has been replaced with acetone, the next step is to infiltrate with
plastic.  This will allow the tissue to be embedded and then cut using the
Ultramicrotomes.  This will also take a few steps that still incorporate
various mixtures of acetone and plastic.

\subsection{Acetone \& Plastic}\label{acetone mix one}
The first step is to remove the acetone from the vial using a micropipette and
replacing it with a 1:1 mixture of acetone and plastic.  Again, as mentioned
before, the vial does not need to be filled up to the brim, just enough to
throughly allow plastic to infiltrate the tissue.

\subsubsection{Agitation}
Put the sample vials in the rotating agitator again for one hour.

\subsection{Acetone \& Plastic Overnight Option**}\label{overnight option}
If you are to finish the process for the day and return the next, then perform
the following option, if not skip to section~\ref{continuing option}.  First
remove the 1:1 mixture from section~\ref{acetone mix one} and replace with a
3:1 mixture of plastic to acetone and let it sit overnight.

\subsection{Acetone \& Plastic}\label{continuing option}
If you are to finish the process the same day then skip section~\ref{overnight
option}.  First remove the 1:1 mixture from section~\ref{acetone mix one} and
replace with a 3:1 mixture of plastic to acetone.

\subsubsection{Agitation}
Put the sample vials in the rotating agitator again for one hour.

\subsection{Pure Plastic}\label{plastic}
First remove the 3:1 mixture from either section~\ref{overnight option} or
\ref{continuing option} and replace with pure plastic.

\subsubsection{Agitation}
Put the sample vials in the rotating agitator again for one hour.
\subsubsection{Vacuum}
Place all of the vials with the lids removed inside the vacuum chamber.  Turn
the pump on to remove air from the chamber.  This will remove all air from the
samples that has been embedded inside the tissue and will allow the
infiltration of plastic to fully take affect.  Let the samples sit inside the
vacuum chamber for one hour.
  
\subsection{Pure Plastic}\label{plastic 2}
Remove the pure plastic from section~\ref{plastic} and replace with pure
plastic again.

\subsubsection{Agitation}
Put the sample vials in the rotating agitator again for one hour.

\subsubsection{Vacuum}
Place all of the vials with the lids removed inside the vacuum chamber.  Turn
the pump on to remove air from the chamber.  This will remove all air from the
samples that has been embedded inside the tissue and will allow the
infiltration of plastic to fully take affect.  Let the samples sit inside the
vacuum chamber for one hour.

\section{Embedding}
The next step is to embed the plasticized tissue into the mold.  Before
forgoing with this process, a list of all of the specimens will need to be
created on Excel to print and cut out.  For example if there are five specimens
in the same vial, make a list of sample names with the specimen ID (A),
specimen ID (B), ... specimen ID (E).  Next, grab a razor blade and a wooden
stir stick.  Simply use the razor blade to shave away wood from the stir stick
to make a flat surface.  The flat surface will be used to transfer specimens
from the vials to the mold.  Place the printed out label inside the mold and
set the mold inside the oven to let it bake the specimens to cure the plastic.  
