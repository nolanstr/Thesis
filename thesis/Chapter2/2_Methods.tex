%%% -*-LaTeX-*-
\documentclass[../Dissertation]{subfiles}

\doublespacing
\graphicspath{{Chapter2/media/}{media/}} % Graphics path for images

\begin{document}

\section{Section}\label{s:Section1}
\vspace{-1ex}
\subsection{Subsection}\label{ss:Subsection1}
    \lipsum[1-2] \cref{fig:Figure_1,tab:Table_1} \lipsum[66] 
    \cite{Feltgen_2014}.

    %--------------------------------%
    \afterpage{
    \begin{figure}[p]
        \centering
        \begin{subfigure}[b]{0.33\textwidth}
            \centering
            \includegraphics[width=\linewidth]
            {example-image-a} % File path to graphics
            \subcaption{\label{fig:a}}
        \end{subfigure}% `%' indicates no space between figures
        \begin{subfigure}[b]{0.33\textwidth}
            \centering
            \includegraphics[width=\linewidth]
            {example-image-b} % File path to graphics
            \subcaption{\label{fig:b}}
        \end{subfigure}% `%' indicates no space between figures
        \begin{subfigure}[b]{0.33\textwidth}
            \centering
            \includegraphics[width=\linewidth]
            {example-image-c} % File path to graphics
            \subcaption{\label{fig:c}}
        \end{subfigure}% `%' indicates no space between figures
        \caption{Example figure with three side-by-side images with no space
        between each figure.  \subref{fig:a} \ldots \subref{fig:b} \ldots
        \subref{fig:c} \ldots \lipsum[75]}
        \label{fig:Figure_1}
    \end{figure}
    \clearpage}
    %--------------------------------%
    \afterpage{
    \begin{table}[p]
        \centering
        \caption{Example table 1. \ldots \lipsum[75]}
        \begin{tabularx}{0.5\textwidth}{XXX}
            \toprule
            $\alpha$  & $\beta$  & $\gamma$  \\
            \midrule
            0         & 2        & 4         \\
            1         & 3        & 5         \\
            \bottomrule
        \end{tabularx}
        \label{tab:Table_1}
    \end{table}
    \clearpage}
    %--------------------------------%
    
    The \hologo{LaTeX} code used to create \cref{fig:Figure_1} is in
    \cref{code:Figure_1}.  
    
    \codeFromFile{LaTeX}
        {./Chapter2/code/LaTeX/Figure1.tex}
        {\ldots \hologo{LaTeX} to display a figure with one three side-by-side images and no space.}
        {Figure_1}
        {\footnotesize}
        {latexcodebg}
        {default}
    
    \lipsum[1-2]
    
    The \hologo{LaTeX} code used to create \cref{tab:Table_1} is in
    \cref{code:Table_1}.  
    
    \codeFromFile{LaTeX}
        {./Chapter2/code/LaTeX/Table1.tex}
        {\ldots \hologo{LaTeX} to display a table with three columns and three rows.}
        {Table_1}
        {\footnotesize}
        {latexcodebg}
        {default}
    
\subsubsection{Cross Reference Label Definition}\label{sss:labeling}
    To cross reference anything in \hologo{LaTeX} you need to define a \hl{label}.
    For example, \emph{Chapters}, \emph{Sections}, \emph{Subsections},
    \emph{SubSubSections}, \emph{Paragraphs}, \emph{Figures}, and \emph{Tables}
    can be labeled in the following manner:
    
    {\singlespacing
    \begin{itemize}
        \setlength\itemsep{0em}
        \item \textbf{A Chapter:}
            \begin{center}`\mintinline{LaTeX}{\chapter{Chapter}\label{chp:1}}' \end{center}
        \item \textbf{A Section:}
            \begin{center}`\mintinline{LaTeX}{\section{Section}\label{s:chp1_overview}}' \end{center}
        \item \textbf{A SubSection:}
            \begin{center}`\mintinline{LaTeX}{\subsection{SubSection}\label{ss:Subsection1}}' \end{center}
        \item \textbf{A SubSubSection:}
            \begin{center}`\mintinline{LaTeX}{\subsubsection{SubSubSection}\label{sss:sss1}}' \end{center}
        \item \textbf{A Paragraph:}
            \begin{center}`\mintinline{LaTeX}{\paragraph{Paragraph}\label{p:paragraph1}}' \end{center}
        \item \textbf{Inside a figure environment:}
            \begin{center}`\mintinline{LaTeX}{\label{fig:figure_1}}'
            \end{center}
        \item \textbf{Inside a table environment:}
            \begin{center}`\mintinline{LaTeX}{\label{tab:table_1}}'
            \end{center}
    \end{itemize}}

\subsubsection{Cross Referencing}\label{sss:cross_referencing}
\paragraph{Document sections}
    To cross reference \emph{Chapters}, \emph{Sections}, \emph{Subsections},
    \emph{SubSubSections}, \emph{Paragraphs}, \emph{Figures}, \emph{Tables},
    \emph{References}, and \emph{Codes} use the following \hologo{LaTeX} commands from the
    \href{https://ctan.org/pkg/cleveref?lang=en}{\texttt{cleveref}} package:
    
    {\singlespacing
    \begin{itemize}
        \setlength\itemsep{0em}
        \item \textbf{A Chapter:}
            \begin{center}`\mintinline{LaTeX}{\cref{chp:1}}'
        $\longrightarrow$ \cref{chp:1}. \end{center}
        \item \textbf{A Section:}
            \begin{center}`\mintinline{LaTeX}{\cref{s:chp1_overview}}'
            $\longrightarrow$ \cref{s:chp1_overview}.\end{center}
        \item \textbf{A SubSection:}
            \begin{center}`\mintinline{LaTeX}{\cref{ss:Subsection1}}'
            $\longrightarrow$ \cref{ss:Subsection1}.\end{center}
        \item \textbf{A SubSubSection:}
            \begin{center}`\mintinline{LaTeX}{\cref{sss:cross_referencing}}'
            $\longrightarrow$ \cref{sss:cross_referencing}.\end{center}
        \item \textbf{A Paragraph:}
            \begin{center}`\mintinline{LaTeX}{\cref{p:paragraph_1}}'
            $\longrightarrow$ \cref{p:paragraph_1}.\end{center}
        \item \textbf{Multiple \emph{Chapters}, \emph{Sections}, \emph{SubSections},
            \emph{SubSubSections}} \\
            \begin{center}`\mintinline[breaklines=True]{LaTeX}{\cref{chp:1,s:chp1_overview,ss:Subsection1,sss:cross_referencing}}'
            $\longrightarrow$
            \cref{chp:1,s:chp1_overview,ss:Subsection1,sss:cross_referencing}.\end{center}
            \emph{with no \hl{space} between each item being referenced.}
    \end{itemize}}
    
\paragraph{References/Citations/Bibliography}
    Use a citation manager to load your references and export a
    \texttt{bibliography.bib} file.  Include this \texttt{bibliography.bib}
    file in your main dissertation file using the
    \mintinline{LaTeX}{\addbibresource{\subfix{Chapter2/bib_files/bibliography.bib}}}
    before the command:  \mintinline{LaTeX}{\begin{document} ... \end{document}}.  It is
    recommended to use \href{https://www.mendeley.com/}{Mendeley} for their
    ease of adding references and exporting \texttt{bibliography.bib} files.
    \href{https://chrome.google.com/webstore/detail/mendeley-web-importer/dagcmkpagjlhakfdhnbomgmjdpkdklff?hl=en}{Mendeley
    Web Importer} also has a Google Chrome extension where you can add
    references via the web browser.
    
    {\singlespacing
    \begin{itemize}
            \item Single references from a \texttt{bibliography.bib} file
            \begin{center}`\mintinline{LaTeX}{\autocite{Feltgen_2014}}' $\longrightarrow$ \autocite{Feltgen_2014}
            \end{center}
            \item Multiple references (\emph{2}) from a \texttt{bibliography.bib} file
            \begin{center}`\mintinline{LaTeX}{\autocite{Gandorfer_2001, Feltgen_2014}}' $\longrightarrow$ \autocite{Gandorfer_2001, Feltgen_2014}
            \end{center}
            \item Multiple references (\emph{3+}) from a \texttt{bibliography.bib} file
            \begin{center}`\mintinline{LaTeX}{\autocite{Fivgas_2001, Gandorfer_2001, Feltgen_2014}}' $\longrightarrow$ \autocite{Fivgas_2001, Gandorfer_2001, Feltgen_2014}
            \end{center}
    \end{itemize}}
    
\paragraph{Figures}
    
    {\singlespacing
    \begin{itemize}
        \setlength\itemsep{0em}
        \item A single figure:  \begin{center} `\mintinline{LaTeX}{\cref{fig:Figure_1}}'
            $\longrightarrow$ \cref{fig:Figure_1}.  \end{center}
        \item Multiple figures:
            \begin{center} `\mintinline{LaTeX}{\cref{fig:Figure_1,fig:Figure_2}}'
            $\longrightarrow$ \cref{fig:Figure_1,fig:Figure_2}  \end{center}
        \emph{with no \hl{space} between each item being referenced.}
    \end{itemize}}
    
\paragraph{Tables}
    
    {\singlespacing
    \begin{itemize}
        \setlength\itemsep{0em}
        \item A single Table:  \begin{center} `\mintinline{LaTeX}{\cref{tab:Table_1}}'
            $\longrightarrow$ \cref{tab:Table_1}.  \end{center}
        \item Multiple Tables:
            \begin{center} `\mintinline{LaTeX}{\cref{tab:Table_1,tab:Table_2}}'
            $\longrightarrow$ \cref{tab:Table_1,tab:Table_2}  \end{center}
        \emph{with no \hl{space} between each item being referenced.}
    \end{itemize}}
    
\paragraph{Figures and Tables}
    
    {\singlespacing
    \begin{itemize}
        \setlength\itemsep{0em}
        \item A Figure and a Table:
            \begin{center} `\mintinline{LaTeX}{\cref{fig:Figure_1,tab:Table_1}}'
            $\longrightarrow$ \cref{fig:Figure_1,tab:Table_1}.   \end{center}
        \emph{with no \hl{space} between each item being referenced.}
    \end{itemize}}
    
\paragraph{Codes}
    
    {\singlespacing
    \begin{itemize}
        \setlength\itemsep{0em}
        \item A single code:  \begin{center} `\mintinline{LaTeX}{\cref{code:Figure_1}}'
            $\longrightarrow$ \cref{code:Figure_1}.  \end{center}
        \item Multiple codes:
            \begin{center} `\mintinline{LaTeX}{\cref{code:Figure_1,code:Figure_2}}'
            $\longrightarrow$ \cref{code:Figure_1,code:Figure_2}  \end{center}
        \emph{with no \hl{space} between each item being referenced.}
    \end{itemize}}

\section{Figure and Table Placement}
    Due to the strict requirement of figure/table placement, it is recommended
    to have figure/tables placed immediately after they are referenced on a
    separate page.  We are using the \hologo{LaTeX} commands from the
    \href{https://ctan.org/pkg/afterpage?lang=en}{\texttt{afterpage}} package
    to do this.  Careful placement in the text needs to be considered to ensure
    that there are not additional pages of text before the figure/table is
    placed.  In some rare instances the figure/table will need to be coded in a
    previous paragraph/section to have the display be on the subsequent page
    after the first mention.  The \hologo{LaTeX} code used to create
    \cref{fig:Figure_1} is in \cref{code:Figure_1}.  
    
    If the figure/tables are to instead be placed in the text at either the top
    or bottom of the page then the placement option \hl{\texttt{[tbp]}} forces
    the figure/table to be either placed at the `top', `bottom', or `on a
    separate page centered vertically'.  The location of the figure/table in
    the text will have to be shifted such that the figure shows up immediately
    after the figure/table is referenced in the text.  \hologo{LaTeX} will try and
    find the best location to minimize white space.  In doing so, sometimes the
    figure/tables will \emph{float} in a less desirable location.  Hence, why
    it is suggested to use the method described above using the
    \href{https://ctan.org/pkg/afterpage?lang=en}{\texttt{afterpage}} package.

\subsection{Table Creation}
    \hologo{LaTeX} can be a bit tricky when it comes to tables.  Therefore, it is
    recommended to use the two following methods to easily create your table
    for publication.  
    
    \begin{enumerate}
        \item \href{https://www.tablesgenerator.com/}{\texttt{tablesgenerator}}
        \item
            \href{https://ctan.org/pkg/excel2latex?lang=en}{\texttt{Excel2\hologo{LaTeX}}}
            package:  Making tables in \hologo{LaTeX} can be tedious, especially if
            some columns are calculated. This converter allows you to write a
            table in Excel instead, and export the current selection as
            \hologo{LaTeX} markup which can be pasted into an existing \hologo{LaTeX}
            document, or exported to a file and included via the
            \mintinline{LaTeX}{\input} command.
    \end{enumerate}

\section{Section}
    \lipsum[1-2]
    
\subsection{Subsection}\label{ss:Subsection2}
    \lipsum[1-2] \cref{fig:Figure_2} 

    %--------------------------------%
    \afterpage{
    \begin{figure}[p]
        \centering
        \begin{subfigure}[b]{1.0\textwidth}
            \centering
            \includegraphics[width=\linewidth]
            {example-image-a} % File path to graphics
            \subcaption{\label{fig:d}}
        \end{subfigure} \\[1ex]
        \begin{subfigure}[b]{0.5\textwidth}
            \centering
            \includegraphics[width=0.95\linewidth]
            {example-image-b} % File path to graphics
            \subcaption{\label{fig:e}}
        \end{subfigure}% `%' indicates no space between figures
        \begin{subfigure}[b]{0.5\textwidth}
            \centering
            \includegraphics[width=0.95\linewidth]
            {example-image-c} % File path to graphics
            \subcaption{\label{fig:f}}
        \end{subfigure}
        \caption{Three figures, one on the top and two side-by-side with gaps
        between figures.  \ldots \subref{fig:d} \ldots \subref{fig:e} \ldots
        \subref{fig:f} \ldots \lipsum[75]}
        \label{fig:Figure_2}
    \end{figure}
    \clearpage}
    %--------------------------------%
    
    \lipsum[1-4]
    
    The \hologo{LaTeX} code used to create \cref{fig:Figure_2} is in
    \cref{code:Figure_2}.  
    
    \codeFromFile{LaTeX}
        {./Chapter2/code/LaTeX/Figure2.tex}
        {\ldots \hologo{LaTeX} to display a figure with one image above and two side-by-side images.}
        {Figure_2}
        {\footnotesize}
        {latexcodebg}
        {default}

\subsection{SubSection}\label{ss:Subsection3}
    \lipsum[1-2]

\subsubsection{SubSubSection}\label{sss:sss3}
    \lipsum[1-2]
    
\paragraph{Paragraph1}\label{p:paragraph_1}
    \lipsum[6-7] \cref{fig:Figure_3} 
    
    %--------------------------------%
    \afterpage{
    \begin{figure}[p]
        \centering
        \begin{subfigure}[b]{0.5\textwidth}
            \centering
            \includegraphics[width=\linewidth]
            {example-image-a}
            \subcaption{\label{fig:g}}
        \end{subfigure}%
        \begin{subfigure}[b]{0.5\textwidth}
            \centering
            \includegraphics[width=\linewidth]
            {example-image-b}
            \subcaption{\label{fig:h}}
        \end{subfigure}%
        \caption{Two figures side-by-side.  \ldots \subref{fig:g} \ldots
        \subref{fig:h} \ldots \lipsum[75]} 
        \label{fig:Figure_3}
    \end{figure}
    \clearpage}
    %--------------------------------%
    
    The \hologo{LaTeX} code used to create \cref{fig:Figure_3} is in
    \cref{code:Figure_3}.  
    
    \codeFromFile{LaTeX}
        {./Chapter2/code/LaTeX/Figure3.tex}
        {\ldots \hologo{LaTeX} to display a figure with two side-by-side images.}
        {Figure_3}
        {\footnotesize}
        {latexcodebg}
        {default}
    
    \lipsum[6-7]
    
\paragraph{Paragraph2}\label{par:Paragraph2}
    \lipsum[8-10] \cref{fig:Figure_4} 
    
    %--------------------------------%
    \afterpage{
    \begin{figure}[p]
        \centering
        \includegraphics[width=1.0\textwidth]
        {example-image-a}
        \caption{Single image \ldots \lipsum[75]} 
        \label{fig:Figure_4}
    \end{figure}
    \clearpage}
    %--------------------------------%
    
    The \hologo{LaTeX} code used to create \cref{fig:Figure_4} is in
    \cref{code:Figure_4}.  
    
    \codeFromFile{LaTeX}
        {./Chapter2/code/LaTeX/Figure4.tex}
        {\ldots \hologo{LaTeX} to display a figure of a single image.}
        {Figure_4}
        {\footnotesize}
        {latexcodebg}
        {default}
    
    \lipsum[8-10] \lipsum[66] \cref{fig:Figure_5} \lipsum[66]
    
    %--------------------------------%
    \afterpage{
    \begin{figure}[p]
        \centering
        \includegraphics[width=1.0\textwidth]
        {example-image-16x10}
        \caption{Single image \ldots \lipsum[75]} 
        \label{fig:Figure_5}
    \end{figure}
    \clearpage}
    %--------------------------------%
    
    \lipsum[75]

\subsection{SubSection}\label{ss:Subsection4}
    \lipsum[10-12]

\end{document}
