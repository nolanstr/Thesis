%--------------------------------%
\afterpage{
\begin{figure}[p]
    \centering
    \begin{subfigure}[b]{0.33\textwidth}
        % A
        \centering
        \includegraphics[width=.95\linewidth]
        {example-image-a} % File path to graphics
        \subcaption{\label{fig:A}}
    \end{subfigure}%
    \begin{subfigure}[b]{0.33\textwidth}
        % B
        \centering
        \includegraphics[width=.95\linewidth]
        {example-image-b} % File path to graphics
        \subcaption{\label{fig:B}}
    \end{subfigure}%
    \begin{subfigure}[b]{0.33\textwidth}
        % C
        \centering
        \includegraphics[width=.95\linewidth]
        {example-image-c} % File path to graphics
        \subcaption{\label{fig:C}}
    \end{subfigure}% \\
    
    \begin{subfigure}[b]{0.33\textwidth}
        % D
        \centering
        \includegraphics[width=.95\linewidth]
        {example-image-a} % File path to graphics
        \subcaption{\label{fig:D}}
    \end{subfigure}%
    \begin{subfigure}[b]{0.33\textwidth}
        % E
        \centering
        \includegraphics[width=.95\linewidth]
        {example-image-b} % File path to graphics
        \subcaption{\label{fig:E}}
    \end{subfigure}%
    \begin{subfigure}[b]{0.33\textwidth}
        % F
        \centering
        \includegraphics[width=.95\linewidth]
        {example-image-c} % File path to graphics
        \subcaption{\label{fig:F}}
    \end{subfigure}%
    \caption{This image contains six subfigures.  \ldots
    (\subref{fig:A}-\subref{fig:C}) \ldots \subref{fig:A}.  \dots
    \subref{fig:B}, \ldots \subref{fig:C}.  \ldots.  \ldots
    (\subref{fig:D}-\subref{fig:F}), \dots \subref{fig:D}, \ldots
    \subref{fig:E}.  \ldots \subref{fig:F}.  Example image \ldots
    \lipsum[75]}
    \label{fig:ABCDEF}
\end{figure}
\clearpage}
%--------------------------------%