\author            {Full Name}
\title             {Thesis/Dissertation Title}
\thesistype        {dissertation} % thesis/dissertation - variable to switch between 3 and 5 committee members
\degreeAchieved    {Doctor of Philosophy}
\department        {Department of xx}
\departmentLink    {}
\submitdate        {Month Year}
\copyrightyear     {\the\year{}}

\abstractString    {The usability of dissertation abstracts depends largely on
their content. Many journals within the medical community have settled on a
seven sentence structure, which is also gaining acceptance in the social
sciences, education and business. In it, the purpose of the study and
methodological choices are outlined succinctly, allowing the reader or
researcher to quickly scan and evaluate a number of studies to easily choose
ones that meet their particular demands. The structure contains variations on
the following seven sentence stems: "The purpose of this study is...." "The
scope of this study...." "The methodology...." "The Findings..." "Conclusions
reached are ..." "Limitations of this study include...." "This study
contributes...." Abstracts of dissertation proposals contain the same seven
concepts, substituting data collection and analysis in place of findings and
conclusions. Abstracts are limited in the United States by the UMI to 350
words.\\\indent More info here.}

\dedication    {Most books at the library will have a dedication page.
Normally, this page includes quotes like "For my mother" or "For Lucy who never
gave up on me." A dissertation dedication is the same concept. In this part of
the dissertation, the student must use a sentence or a paragraph to dedicate
their text. They may want to use the dedication to recognize an individual who
inspired them to go to college or someone who helped with the dissertation.
Dedicating the dissertation to someone is a way to honor them. After putting so
much work into this paper, it is a chance for the student to recognize the
people who influenced the process.}

\frontispiece    {example-image-a} % 

\epigraphQuote    {``Quote"} 
\epigraphAuthor    {Famous Individual}

\Acknowledgement{The dissertation acknowledgements section is where you thank
those who have helped and supported you during the research and writing
process. This includes both professional and personal acknowledgements. \ldots
\\\indent More acknowledgement info can be found here:
\burl{https://www.scribbr.com/dissertation/acknowledgements/}.}

\NotationAndSymbols{
\begin{tabularx}{\textwidth}{llX}
    \toprule
    x     & \emph{Var} &- the variable `x'. \\
    y     & \emph{y} &- the variable `y'. \\
    m     & \emph{Slope} &- The slope is one of the essential characteristics of a line and helps us measure the rate of change. The slope of a straight line is the ratio of the change in $y$ to the change in $x$, also called the rise over run. \\
    $\bvec{F}$     & \emph{Force} &- Force vector. \\
    $\pi$     & \emph{Pi} &- The number $\pi$ is a mathematical constant. It is defined as the ratio of a circle's circumference to its diameter, and it also has various equivalent definitions. It appears in many formulas in all areas of mathematics and physics. \\
    \bottomrule
\end{tabularx}%
}

\approvaldepartment     {Dept.} % Your department goes here

\graduateDean           {Graduate School Dean} % Dean's name
\departmentChair        {Department Chair}
\departmentChairTitle   {Chair}
\deptmentCollegeSchool  {Department} % College/Department/school

\committeeChair         {Graduate Advisor} % Graduate Advisor
\advisorTitle           {Associate Professor, Associate Chair}

\committeeChair         {Graduate Advisor} % Graduate Advisor
\advisorTitle           {Associate Professor, Associate Chair}

\committeeMemberII      {Committee member 2}
\committeeMemberIII     {Committee member 3}
\committeeMemberIIII    {Committee member 4}
\committeeMemberIIIII   {Committee member 5}

\chairDateApproved                  {mm/dd/\the\year{}}
\committeeMemberIIDateApproved      {mm/dd/\the\year{}}
\committeeMemberIIIDateApproved     {mm/dd/\the\year{}}
\committeeMemberIIIIDateApproved    {mm/dd/\the\year{}}
\committeeMemberIIIIIDateApproved   {mm/dd/\the\year{}}

