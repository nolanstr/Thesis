%%% -*-LaTeX-*-
\documentclass[../Dissertation]{subfiles}

\doublespacing

\begin{document}

\chapter{Code Script Examples}\label{AppendixA}
    
    % If there is no text between the chapter and the first section, then we 
    % need to reduce the spacing with a negative skip.  Otherwise, comment out
    % the following line
    \vspace{-0.5\UofUDoubleSpace}
    
\section{Example Cross product codes}
    Find cross products using
    \cref{code:ex1,code:ex2,code:ex3,code:ex4}.  
    
\subsection{\texttt{c}}
    \codeFromFile{c}
    {\subfix{code/c/HW3-4_58.c}}
    {\ldots \texttt{c} to solve cross products.}
    {ex1}
    {\footnotesize}
    {ccodebg}
    {default}

\subsection{\texttt{Fortran}}
    \codeFromFile{fortran}
    {\subfix{code/fortran/HW3-Problem_2.f}}
    {\ldots \texttt{Fortran} to solve cross products.}
    {ex2}
    {\footnotesize}
    {fortrancodebg}
    {default}

\subsection{\texttt{Matlab}}
    \codeFromFile{matlab}
    {\subfix{code/matlab/HW3-2_cross_product.m}}
    {\ldots \texttt{Matlab} to solve cross products.}
    {ex3}
    {\footnotesize}
    {matlabcodebg}
    {default}

\subsection{\texttt{Python}}
    \codeFromFile{python}
    {\subfix{code/python/HW3-2_cross_Product.py}}
    {\ldots \texttt{Python} to solve cross products.}
    {ex4}
    {\footnotesize}
    {pythoncodebg}
    {default}

\section{Additional Example codes}

\subsection{\texttt{Matlab}}
    \codeFromFile{matlab}
    {\subfix{code/matlab/HW3-problem_1.m}}
    {\ldots \texttt{Matlab} to solve a problem.}
    {HW3-problem_1}
    {\footnotesize}
    {matlabcodebg}
    {default}
    
\section{Code Highlighting Using
\href{https://ctan.org/pkg/minted?lang=en}{\texttt{minted}}}
\subsection{Highlighting Function}
    To use the code highlighting function that renders scripts for publication
    with line numbers/language specification/file path and more, please see the
    following function:
    
    {\singlespacing
    \begin{minted}{LaTeX}
    % The "\codeFromFile" function is used in the following manner:
    \codeFromFile
        {language}        % Programming language
        {\subfix{path}}   % File path
        {Header}          % Script heading info
        {label}           % LaTeX label for cross referencing
        {fontsize}        % Fontsize
        {backgroundcolor} % Text background color
        {mintedStyle}     % Minted text style (default)
    \end{minted}
    }

\subsection{Code Highlighting Example}
    Here is the text to display the code used to highlight \hologo{LaTeX} code in
    \cref{code:exampleCode}.
    
    {\singlespacing
    \begin{minted}{LaTeX}
    \section{Scripting language to call scripts}
    \subsection{\texttt{\hologo{LaTeX}}}
        \codeFromFile{LaTeX}
        {\subfix{code/LaTeX/exampleCode.tex}}
        {\ldots \hologo{LaTeX} to call other scripts.}
        {exampleCode}
        {\footnotesize}
        {latexcodebg}
        {default}
    \end{minted}
    }
    
    \begin{itemize}
        \item Lines \texttt{1-8} in \cref{code:exampleCode} refer to
            \cref{code:ex1}
        \item Lines \texttt{10-17} in \cref{code:exampleCode} refer to
            \cref{code:ex2}
        \item Lines \texttt{19-26} in \cref{code:exampleCode} refer to
            \cref{code:ex3}
        \item Lines \texttt{28-35} in \cref{code:exampleCode} refer to
            \cref{code:ex4}
    \end{itemize}


\subsection{\texttt{\hologo{LaTeX}} Script}
    \codeFromFile{LaTeX}
    {\subfix{code/LaTeX/exampleCode.tex}}
    {\ldots \hologo{LaTeX} to call other scripts using the
    \texttt{codeFromFile} function.} % \mintinline{LaTeX}{\codeFromFile}
    {exampleCode}
    {\footnotesize}
    {latexcodebg}
    {default}

\end{document}
